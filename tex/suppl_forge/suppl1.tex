\documentclass[11pt,landscape]{report}
\usepackage{graphicx}
\usepackage{amssymb}
\usepackage[colorlinks]{hyperref}
\usepackage{tocloft}
\usepackage{xspace}



\textwidth = 10 in
\textheight = 7.0 in
\oddsidemargin = -0.5 in
\evensidemargin = -0.5 in
\topmargin = -0.5 in
\headheight = 0.0 in
\headsep = 0.0 in
\footskip = 0.65 in
\parskip = 0.2in
\parindent = 0.0in


\newcommand{\snptype}{SNP~Type\textsuperscript{{\tiny TM}}}
%% Some pretty etc.'s, etc...
\newcommand{\cf}{{\em cf.}\xspace }
\newcommand{\eg}{{\em e.g.},\xspace }
\newcommand{\ie}{{\em i.e.},\xspace }
\newcommand{\etal}{{\em et al.}\ }
\newcommand{\etc}{{\em etc.}\@\xspace}


% this stuff just has to be in here so it doesn't choke 
% on the stuff that I put in for typesetting the submission version.
\newcommand{\AuthorAddresses}{}
\newcommand{\KeyWords}{}
\newcommand{\CorrespondingAuthor}{}
\newcommand{\RunningTitle}{}



% get TOC spacing onto one page
\setlength\cftparskip{2pt}
\setlength\cftbeforepartskip{2pt}

%!TEX root = flock-comment-main.tex


%%%%% THIS IS THE SECTION WHERE THE AUTHOR PUTS IN ALL OF THEIR TITLE AND AFFILIATION
%%%%% INFORMATION AND  A FEW OTHER THINGS
\newcommand{\myTitle}{Genetic and individual assignment of tetraploid green sturgeon with  SNP assay data}
\title{\myTitle}

% redefine this to make author list. Note affiliation symbols are done manually.
% All caps tends to look better
\newcommand{\myAuthors}{Eric C. Anderson$^{*,\S}$, Thomas C. Ng$^\dagger$, Eric D. Crandall$^{*,\ddag}$, and
John Carlos Garza$^{*}$}
%\author{AUTHOR ONE$^{*}$ and PATRICK D. BARRY$^\dagger$}


% redefine to make the affiliation list.  Note symbols are done manually
\newcommand{\myAffiliations}{$^*$Fisheries Ecology Division, 
    Southwest Fisheries Science Center, National Marine Fisheries Service, NOAA,
    110 Shaffer Road,
    Santa Cruz, CA 95060, USA, $^\dagger$Biomolecular Engineering, UCSC. $\ddag$Current address: Division of
    Science and Environmental Policy, California State University, Monterey Bay.}
\renewcommand{\AuthorAddresses}{\myAffiliations}

\renewcommand{\KeyWords}{Unsupervised clustering, simulated annealing, software}

\renewcommand{\CorrespondingAuthor}{Eric C. Anderson, Fisheries Ecology Division, Southwest Fisheries Science
Center, 110 Shaffer Road, Santa Cruz, CA 95060. eric.anderson@noaa.gov}


% the email address for the corresponding author
\newcommand{\myEmailAddress}{eric.anderson@noaa.gov}
\newcommand{\myEmailFootnote}{$^\S$}

% here you can put your very own copyright notice
\newcommand{\myCopyright}{\copyright US Federal Government work in the public domain in the USA}

% here you can put a running title (a short title that goes on the left of the 
% even pages)
\newcommand{\myRunningTitle}{Assignment of polyploids with SNPs}
\renewcommand{\RunningTitle}{\myRunningTitle}

% and here you put the running author (short listing of authors for the
% upper right header on the odd pages)
\newcommand{\myRunningAuthor}{Anderson et al.}

%%%% DONE WITH AUTHOR/TITLE/ETC INFORMATION DEFINITION
  % Put all the author title stuff in there

\title{Supplement \# 1 to Article:\\
\myTitle\\
\mbox{}\\
{\em Scatter figures like Figure~1 for all SNP assays } }




\begin{document}
\maketitle
\section*{Overview/Orientation}
This supplement contains all the scatter plots like those that appear in Figure~1
of the paper to which this is a supplement.  

Figures 1, 3, 5, and 7 in this supplement correspond to the plots in Figure~1{\em a} in the paper.

Figures 2, 4, 6, 8 in this supplement correspond to the plots in Figure~1{\em b} in the paper.

The caption from Figure 1 in the original paper applies to these figures and is paraphrased here as a single
caption for all of the supplemental figures:
\begin{quotation}
Raw \snptype{} fluorescence intensities at all 96 assays typed upon the four
scoring-development chips. Each panel
shows the results at four chips for one locus.  The numeral in the upper right gives the number
of genotype categories that are scored at the locus.  Each point represents an individual's fluorescence
intensities at two dyes. {\bf Figures~1,~3,~5,~7}: Individuals are colored according to chip and dashed line
segments connect the intensities of individuals typed on different chips. {\bf Figures~2,~4,~6,~8}: Individuals are colored
according to the genotype category to which they have been scored.  Category~0 (colored light gray) is a ``no call'', \ie the individual
is recorded as having missing data at the locus. \end{quotation}

Save a tree. Please don't print this out.  View it on your computer.

\newpage

\begin{figure}
\includegraphics*[width = \textwidth]{images/plate_x_y_with_num_clusts_1.pdf}
\caption{Colored by chip or chip-pair. See description in the Overview/Orientation.}
\end{figure}

\begin{figure}
\includegraphics*[width = \textwidth]{images/plate_x_y_by_final_genotype_plates_combined1.pdf}
\caption{Colored by called genotype. See description in the Overview/Orientation.}
\end{figure}




\begin{figure}
\includegraphics*[width = \textwidth]{images/plate_x_y_with_num_clusts_2.pdf}
\caption{Colored by chip or chip-pair. See description in the Overview/Orientation.}
\end{figure}

\begin{figure}
\includegraphics*[width = \textwidth]{images/plate_x_y_by_final_genotype_plates_combined2.pdf}
\caption{Colored by called genotype. See description in the Overview/Orientation.}
\end{figure}







\begin{figure}
\includegraphics*[width = \textwidth]{images/plate_x_y_with_num_clusts_3.pdf}
\caption{Colored by chip or chip-pair. See description in the Overview/Orientation.}
\end{figure}

\begin{figure}
\includegraphics*[width = \textwidth]{images/plate_x_y_by_final_genotype_plates_combined3.pdf}
\caption{Colored by called genotype. See description in the Overview/Orientation.}
\end{figure}







\begin{figure}
\includegraphics*[width = \textwidth]{images/plate_x_y_with_num_clusts_4.pdf}
\caption{Colored by chip or chip-pair. See description in the Overview/Orientation.}
\end{figure}

\begin{figure}
\includegraphics*[width = \textwidth]{images/plate_x_y_by_final_genotype_plates_combined4.pdf}
\caption{Colored by called genotype. See description in the Overview/Orientation.}
\end{figure}








\end{document}